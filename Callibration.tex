\documentclass[a4paper,12pt]{article}
\usepackage[a4paper, total={6.5in, 10in}]{geometry}
\usepackage{gensymb}

\begin{document}

\title{Analog Theodolite Vertical Index and Horizontal Collimation Calibration Procedure}
\author{Micah Johnson}
\date{\today}
\maketitle

\section*{Overview}
The analog Kern DKM2-AEC theodolite may be used in the optical alignment of the ESIS and MOSES instruments. Before the theodolite should be used for alignment, its Vertical Index and Horizontal Collimation error should be checked and corrected if not properly celebrated.

\section*{Vertical Index Calibration}
	\begin{enumerate}
	\item Level theodolite using Leveling Knobs. Make sure it is level in all directions by rotating the theodolite 120\degree horizontally and re-leveling multiple times.
	\item Connect the battery to the wires that power the blue LED, illuminating the autocollimation attachment.
	\item Set up flat mirror in a tip/tilt kinematic mount  at least 4 feet from the theodolite.
	\item Adjust theodolite until cross-hairs are centered on the surface of the mirror.
	\item Change focus to the reflected theodolite image. Theodolite or user should be in focus.
	\item Adjust tip/tilt of mirror so reflected image is directly back into the theodolite lens cavity.
	\item Focus theodolite at infinity or until return cross-hairs are clearly visible.
	\item Adjusting the tip tilt of the theodolite, center the return cross-hairs with the original cross-hairs.a
	\item Adjust the Lighting Mirror so all displays are visible in the Reading Eyepiece. Sometimes it is helpful to set up a lamp over the Lighting Mirror.
	\item Look through the Reading Eyepiece, and twist the micrometer knob until the single vertical line(s) are centered between the 2 close vertical lines in the vertical display.
	\item Read the vertical angle as explained in figure 13 of the theodolite user manual and enter the value into the CalibrationCalculator.xlsm excel sheet.
	\item Rotate theodolite 180\degree in both horizontal and vertical axes and repeat steps 4-11 excluding step 6. Moving the mirror at this point will ruin the experiment.
	\item Move the Micrometer Knob to the correct angle, without actuall adjusting the position of the theodolite.
	\item Adjust the screw shown in figure 25 of the user manual until the single vertical line(s) are centered between the 2 close vertical lines in the vertical display.
	\item Repeat procedure until desired error is obtained.
	\end{enumerate}

\section*{Horizontal Collimation Calibration}
    This is the same procedure as the Vertial Index Calibration, but dials instead of a screw is used to correct the error. This can be done at the same time as the Vertial Index Calibration if desired.
    \begin{enumerate}
    \item Level theodolite using Leveling Knobs. Make sure it is level in all directions by rotating the theodolite 120\degree horizontally and re-leveling multiple times.
	\item Connect the battery to the wires that power the blue LED, illuminating the autocollimation attachment.
	\item Set up flat mirror in a tip/tilt kinematic mount  at least 4 feet from the theodolite.
	\item Adjust theodolite until cross-hairs are centered on the surface of the mirror.
	\item Change focus to the reflected theodolite image. Theodolite or user should be in focus.
	\item Adjust tip/tilt of mirror so reflected image is directly back into the theodolite lens cavity.
	\item Focus theodolite at infinity or until return cross-hairs are clearly visible.
	\item Adjusting the tip tilt of the theodolite, center the return cross-hairs with the original cross-hairs.
	\item Adjust the Lighting Mirror so all displays are visible in the Reading Eyepiece. Sometimes it is helpful to set up a lamp over the Lighting Mirror.
	\item Look through the Reading Eyepiece, and twist the micrometer knob until the single vertical line(s) are centered between the 2 close horizontal lines in the horizontal display.
	\item Read the horizontal angle as explained in figure 13 of the theodolite user manual and enter the value into the CalibrationCalculator.xlsm excel sheet.
	\item Rotate theodolite 180\degree in both horizontal and vertical axes and repeat steps 4-11 excluding step 6. Moving the mirror at this point will ruin the experiment.
	\item Move the Micrometer Knob to the correct angle, without actually adjusting the position of the theodolite.
	\item Adjust the Circle Orienting gear dial until the single vertial line(s) are centered between the 2 close vertial lines in the horizontal display.
	\item Repeat procedure until desired error is obtained.
    \end{enumerate}

\end{document}
